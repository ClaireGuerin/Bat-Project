\documentclass[12pt]{article}
\usepackage{setspace}
\doublespacing
\usepackage{graphicx}
\usepackage[export]{adjustbox}
\usepackage{caption}
\usepackage{scicite}
\usepackage[space]{grffile}

\usepackage{titlesec}%
\titleformat{\section}%
  [hang]% <shape>
  {\normalfont\bfseries\Large}% <format>
  {}% <label>
  {0pt}% <sep>
  {}% <before code>
\renewcommand{\thesection}{}% Remove section references...
\renewcommand{\thesubsection}{\arabic{subsection}}%... from subsections

\usepackage[margin=1in]{geometry}

\newenvironment{sciabstract}{%
\begin{quote} \bf}
{\end{quote}}

\newcounter{lastnote}
\newenvironment{scilastnote}{%
\setcounter{lastnote}{\value{enumiv}}%
\addtocounter{lastnote}{+1}%

\begin{list}%
{\arabic{lastnote}.}
{\setlength{\leftmargin}{.22in}}
{\setlength{\labelsep}{.5em}}}
{\end{list}}

\usepackage{listings}
\usepackage{color}

\definecolor{dkgreen}{rgb}{0,0.6,0}
\definecolor{gray}{rgb}{0.5,0.5,0.5}
\definecolor{mauve}{rgb}{0.58,0,0.82}

\lstset{frame=topline,
  aboveskip=3mm,
  belowskip=3mm,
  showstringspaces=false,
  columns=flexible,
  basicstyle={\small\ttfamily},
  numbers=none,
  numberstyle=\tiny\color{gray},
  keywordstyle=\color{blue},
  commentstyle=\color{dkgreen},
  stringstyle=\color{mauve},
  breaklines=true,
  breakatwhitespace=true,
  tabsize=3
}

\title{Quantifying Potential Sonar Interference in Bat Aggregations}

\author
{Claire Gu\'{e}rin$^{1,2,3\ast}$\\ Thejasvi Beleyur,$^{1}$ Holger Goerlitz$^{1}$\\
\\
\normalsize{$^{1}$Goerlitz Research Group 'Acoustic and Functional Ecology'}\\
\normalsize{Max Planck Institute for Ornithology, Seewiesen, Germany}\\
\normalsize{$^{2}$Ludwig-Maximilian University, Munich, Germany}\\
\normalsize{$^{3}$Erasmus Mundus Master Programme in Evolutionary Biology}\\
\\
\normalsize{$^\ast$E-mail:  claire.guerin@evobio.com}
}

\date{}

\begin{document}

\singlespacing

\maketitle

\begin{sciabstract} 

Chiropters that undergo echolocation face diverse intricacy issues on a sensory level, due to echolocation jamming. One particular circumstance leading to sonar interference is the one that springs when bats fly in groups. In such situation, an individual bat encounters direct sensory interference from its congeners, which also emit echolocation calls. In order to cope with sensory confusion emerging from intraspecific sonar jamming, bats developed various strategies, all related to call structure modification. However, it is still poorly understood to which extent these adaptations enable bats to alleviate sonar interference that arises in situations of relatively close group flight. With the aim of apprehending the impact of change in call inter-pulse interval, we built an individual-based model that reproduces groups of bats flying in a restricted environment, and regularly echolocating. We show that the alteration of inter-pulse interval can lead to an adjustment of the theoretical minimum level of sonar interference, subject to other variables (group size and density, duty cycle...). We hope that this model, suitable to most echolocating bat species, will be used to provide further understanding on the amount of sensory interference that bats are able to overcome in the context of bat aggregations.

\end{sciabstract}

\doublespacing

\section{Introduction}

Coordinated group movement has raised increasing interest for the past two decades, both in the fields of ethology, ecology and evolution, and this, within a broad range of taxa and ecological contexts (among others, fish shoals ~\cite{Lopez2012}, bird flocks ~\cite{Cavagna2010}, mussel beds ~\cite{Liu2014}, and even bacteria ~\cite{Zhang2010}). As typical study case of self-organized behaviour, group motion is of great appeal for statistical physicists as well as theoretical biologists. The coordinated movement of an individual with the flow of its neighbours depends on three components: sensory input ('sense'), neural processing ('analyse') and manoeuvrability ('move'). Individual-based models (IBM) have proved useful in order to discern the theory behind group motion behaviour, and therefore have been extensively applied by theoretician biologists, seeking to understand how interactions between individuals on a small spatial scale drive patterns of organized flux on a group level (see for example ~\cite{Accolla2015}). However, most of these models rely on assumptions concerning the underlying rules that individuals follow, that are not clearly justified by behaviour logic nor empirical studies ~\cite{Lopez2012}. There is a recognized need for models of group movement that are more oriented towards biology. Recent studies have shown the determinant role of sensory cues in the efficiency of animal sensory decisions in regard to movement, and thus, coordination in group ~\cite{Partridge1980,Ballerini2008,Strandburg-Peshkin2013}. Animal species such as echolocating bats that use active sensing of the environment for orientation can be particularly helpful in understanding how and how much the quality of sensory input can impact the capacity of an individual to adapt its movements with regard to others ~\cite{Lin2015,Giuggioli2015}. Indeed, active sensing through echolocation meets quantifiable laws of physics, which allow for relatively accurate accounting of an individual’s perception and possibility of detecting, identifying and locating an object in its environment ~\cite{Rossing2007,Prandoni2008a}. Most bat species (about ¾ of the Chiroptera order) echolocate ~\cite{Surlykke2014}, meaning that they actively scan their surroundings for orientation as well as foraging ~\cite{Schnitzler2003}. Echolocating bats emit echolocation calls or pulses of high frequency (ultrasounds) at determined intervals, and in return, receive echoes from nearby objects as sound is reflected. They almost instantly analyse and interpret these echoes in order to reconstruct an acoustic image of their milieu, often referred to as soundscape. Although they usually do not exhibit swarm behaviour in its most restricted sense, many species of bats are highly social (tens to millions of individuals), and may therefore exhibit collective behaviour (e.g. roosting behaviour) ~\cite{Kunz2005}. These species are able to fly in groups of high densities ~\cite{Betke2007,Gillam2010}, at high speed ~\cite{Theriault2010}, in such circumstances as multiple Daubenton’s bats (Myotis daubentonii) hunting over a water surface ~\cite{Kalko&Schnitzler1989}, or colonies coming in and out of their cave by hundreds ~\cite{Rivers2006,Lee2001}. In such contexts, echolocating bats need to achieve a certain degree of group coordination in order to avoid collisions. Even though several studies have described manoeuvre tactics adopted by bats to physically circumvent their surrounding peers, unfortunately little is known on how bats overcome sonar jamming and sensory confusion due to intraspecific noise. However, as intensity of echoes dramatically decreases due to propagation and partial absorption of sound when encountering a specific object, the detection range of a bat’s sonar beam is limited to several meters in front at best, and less on the sides. Thus, sensory input is of primary importance, as an efficient discrimination of reflected echoes is crucial for manoeuvring within a group of congeners. Despite the fact that in situations of flight in group, a few bat species are known to keep quiet and choose eavesdropping over active echolocation in order to locate their congeners ~\cite{Barclay1982,Chiu2008}, point has been made that bats can and most of the time do detect each other thanks to their biosonar instead. The great flexibility of bats in controlling and regulating parameters such as intensity or shape of the call allows them to develop strategies aiming at reducing potentially destructive interference during their active sensing from peer’s sounds. Some strategies that have been suggested to be embraced by bats when they fly in group are a change in the amplitude, the frequency range ~\cite{Bates2008,Hiryu2010}, the amplitude or the duration of a pulse ~\cite{Amichai2015}, and even though each comes at a cost, it also grants a reduction of overlap between a bat’s own echoes and echoes from others.  It has been suggested that bats modulate the interval between each pulse when they find themselves in a situation where they are subject to sensory jamming induced by echolocating peers ~\cite{Jarvis2013,Petrites2009}. By doing so, they are able to reduce the overlap between returning echoes from their own pulses and sounds generated by other individuals. However it is still unclear in which way pulse rate modification could enhance sonar performance in social settings. With the prospect of untangling this dynamic issue, we developed an individual-based model for multiple bat agents moving in a two-dimensional space, and emitting echolocation calls at different inter-pulse interval. This model, described in the first part of the paper using the ODD protocol recommendations, allows to quantify the time of sound overlap between a bat's echo and calls emitted by its surrounding congeners. We then estimate the resulting confusion between self and non-self sounds based on several indices for measuring sensory interference. We discuss the theoretical impact of inter-pulse interval modulation in echolocation performance in the context of group flight, based on results of specific simulation examples, and assess the advantages and downsides of our model.

\section{Model Description}

\subsection{Overview}
\label{sec:ovv}

The purpose of the model is to quantify the theoretical sonar interference between a bat's own returning echoes and calls generated by conspecifics when groups of flying bats undergo echolocation.
 
\begin{description}

\item [Entities, state variables and scale] Thus, the agents of the model are individual bats. Each agent is defined by a unique identity number, a duty call, a sensory library of perceived sounds and Cartesian coordinates. The duty call of the agent is constant and depends on the state variables call duration and inter-pulse interval (in $seconds$). The location of the agent changes over the course of time and depends on the flight speed (in $meters\times seconds^{-1}$, angle of flight direction (in $radians$) and previous position in space (in $meters$). The individual positions are determined by a superior entity that incorporates the group of bats in the simulation as a whole. This entity is defined by the population size, the inter-individual distances on the vertical and horizontal axes, and the direction homogeneity of the individuals.
This model aims to reproduce a real-world biological scenario in continuous time and space, where  bat individuals fly in an open environment(that is to say, without obstacles of any sort), and undergo echolocation. For computational reasons however, while space is calculated as a continuous variable, time is discretised and integrated over every time step of a simulation.
$t$, $n$ and $\Delta t$ correspond to time in seconds, time step and time resolution in seconds per time step respectively, with $\Delta t=t/n_t$. Thus, a time step $n_t$ integrates the time $t$ in seconds such as $t\in [n_t\times\Delta t;n_t+1\times\Delta t[$. The simulatory space does not satisfy closed-boundaries conditions, which implies that the bat agents in the model can theoretically travel endlessly, as far as the simulation lasts.
The model does not account for environmental input, that is to say that no external forces (temperature, clutter, obstacles and preys) drive the behaviour and dynamics of the agents.

\item [Process overview and scheduling] Simulations are run under Python 2.7 and post-hoc calculations and analysis are produced with R Software 3.3.0. As previously described, time is modelled such as one time step constitutes a continuum over which both continuous processes and discrete events can occur. The general process scheduling can be summarized by the pseudo-code below:

\begin{quote}
Python 2.7
\begin{lstlisting}[language = Python]
for t in range(durationOfSimulation):
	for i in range(population_size):
		Initiate()  #i initiates a call, keeps calling or remains silent
		Erase() #erase from memory calls emitted by i for which the intensity is now too faint to be heard
		Propagate() #propagate every call emitted by i
	for i in range(population_size):
		Sense() #sense calls that are within hearing distance
	for i in range(population_size):
		Move() #displace agent
write output in file
\end{lstlisting}
R 3.3.0
\begin{lstlisting}[language = R]
read.table(file)
for (i in 0:populationSize - 1):
	Echoes() #calculate timing of own echoes
	Filter() #filter out calls and echoes heard by i when it was calling 
	Overlaps() #find overlaps between echoes and echoes and calls
\end{lstlisting}
\end{quote}

During the simulation, state variables are asynchronously updated, as they are immediately assigned a new value as soon as this value is calculated by a process. 

\end{description}

\subsection{Design concepts}

\begin{description}

\item [Basic principles] ... IN CONSTRUCTION

\item [Emergence] When the initial inter-individual distances, the homogeneity of the agents' flight direction and the individual duty cycles are changed, the interference of passive sensing with active sensing of the agents is expected to vary. That is to say, the overlap between perceived calls emitted by congeners and an agent's own returning echoes is expected to fluctuate, which alters the amount of masking of actual information (interpreted from echolocation) by the noise produced by other bats in the surroundings.
The objective is to determine an optimal inter-pulse interval for a bat agent to accurately scrutinize the soundscape through echolocation. Thus, success criteria for achieving echolocation is calculated as the mean number of calls that overlap with echoes, the mean time of such overlap, and the percentage of inter-pulse interval that is free of calls from congeners, over the whole simulation time.  

\item [Adaptation] This model is deterministic, as the duty-cycle of each individual is determined by the modeller beforehand and remains constant throughout a simulation. Thus, the agents to not possess adaptive traits. However, the model is built to ultimately achieve an adaptive model, where inter-pulse intervals are allowed to evolve through a feedback loop based on the quantified sonar interference between self echoes and non-self calls.

\item [Sensing] Bat individuals in the simulated group perceive each other through hearing abilities, limited by their hearing threshold. This threshold is implicitly modelled, as is sound intensity, by determining the maximum distance at which an agent can hear a sound. The individuals have to ways to hear each other, namely through active or passive sensing. Active sensing, based on echolocation, relies on the sounds that a bat hears back from calls that were emitted by the bat itself, and reflected on another bat individual in the from of an echo. Passive sensing consists of hearing the calls that were emitted by congeners. Echoes formed from these non-self calls are neglected. Thus, sensing in the model consists of a network, whose structure is imposed by the modeller when setting the group's state variables of initial positioning and movement homogeneity. The obtention of sensory information by the individuals is explicitly modelled in the Sensing submodel. 

\item [Interaction] The bat individuals in the model interact with each other solely through perception. Indeed, interference in the echolocation process results from an interaction between active perception of an individual, and passive perception imposed on the same individual by its congeners, when the latter perpetrate calls, thus negatively affecting the individual's echolocation process.

\item [Stochasticity] In the real-world, bats do not echolocate all the time. Indeed, echolocation is mainly reserved for foraging, and bats often display other types of calls such as social calls, or even go silent in some cases. The model however is specifically designed to study sonar interference in groups of echolocating bats. As a consequence, the agents are assumed to constantly echolocate. Stochasticity is thus used to reproduce variability in times at which independent individuals produce call pulses. The actual causes for which individual bats incidentally emit calls at various times is not essential to the model, which explains why we chose to haphazardly assign to individuals, times at which they start echolocation. The starting time of the initial call of each individual is pseudo randomly determined at the beginning of the simulation. It is sampled between $t=0$ and $t=IPI + Duration$ in the simulation. 

\item [Collectives] As stated in the \ref{sec:ovv} section, all individual bats belong to a unique aggregation, or group of bats, with inter-individual distances that are maintained constant throughout the simulation, on at least one of the 2 axes on the plane. Unlike agent-based models reproducing aggregation patterns in bird flocks for instance, the collective entity formed by all bats in this model is not an emergent property of individuals, rather defined by the modeller. If the modeller decides that all individuals head towards the same direction, distances between neighbours both the vertical and horizontal axes are constant. Otherwise, in case some of the individuals head towards to opposite direction, distances between neighbours on either one of the axes remains the same, while distances vary with time along the other axis. At $t_0$, the agents are disseminated in space and organized in a square lattice. This way, distances between neighbours on the vertical and horizontal axes, and distances between neighbours on the diagonal, respectively, are equal (~\ref{posmov}). Bat agents' movements describe a straight line with a determined direction, and are determined within a plane surface. Time being discrete, a vector transcribes bat movement during one simulation time step between two consecutive positions. Thus, the coordinates on the x and y axis are calculated following the equations:\\

\begin{math}
	\left\{
		\begin{array}{l}
			x_{n+1}=x_n+\Delta t\times v_{bat}\times\cos(\theta)\\
			y_{n+1}=y_n+\Delta t\times v_{bat}\times\sin(\theta)
		\end{array}
	\right.
\end{math}
\\

Where $v_{bat}$ is the flight speed of the bat (set identical for every agent in a simulation), and $\theta$ is the direction towards which the agent is headed, expressed as the angle formed with the x-axis of the plane.

\begin{center}
\includegraphics[max size={\textwidth}{\textheight}]{posmov.pdf}
\captionof{figure}{Example of space setting of $N=9$ bat agents, positioned in a square lattice at 1 meters intervals along the axes. The coordinates, at $t_1$, of the agent in the centre of the square depend on the direction and distance of the vector in red, which represents the.}
\label{posmov}
\end{center}

\item [Observation] Although individual positions and starts of calls are recorded for all time steps in the simulations, only the sensory data is collected for further analysis, in the form of timing of sound perception. For each individual, the time steps at which a sound is heard - being a call emitted by another individual or the echo emerging from a call emitted by the individual itself - are acquired. We assume that when a bat calls, it is not able to perceive any external sound as the ears and cochlea are temporarily disabled through mechanical physical operation. The sounds that reach the ears of an individual during the simulation, are therefore previously filtered by the times at which the individual itself performs a call, before further sensory analysis. 

\end{description}

\subsection{Initialization}
\label{subsec: init}

At time $t_0$, the simulation environment contains a population or group of bats of size N. The population size is a fixed variable, and therefore does not change throughout a single simulation. The bat agents in the population are initially placed at regular axial intervals $d$ on the plane so that they constitute a square lattice of the form $d\cdot (\sqrt{N}-1)^2$. The axis-aligned square lattice determines the positions of every bat according to the provided coordinates of the individual the closest to the plane origin. Flight speed ($5.5 m\cdot sec^{-1}$) and movement angle ($\theta radians$) are attributed to each agent. Individual call structure is pre-determined by call duration ($0.07 sec$) and IPI ($I sec$). The maximum distance for a bat to hear a sound is calculated based on the sound absorption factor ($\alpha dB\cdot m^{-1}$), the sound source pressure level ($dB$ SPL, reference $20 mPa$ at $10 cm$), and the bat hearing threshold ($\Theta_H dB$ SPL).
Although the model allows for individual differences in every initial parameters previously stated, we decide to maintain homogeneity in flight and call characteristics of the population in every simulation. IPI is the main parameters that varies between simulations, alongside with population size and movement direction.  

The model does not use input data to represent time-varying processes.

\subsection{Submodels}

\subsubsection{Call: 'Initiate', 'Erase', 'Propagate'}
Throughout a simulation, bat agents call at regular inter-pulse intervals (IPIs), defined by $I$, which are set identical for all agents in the simulation. The duration D of call pulses is also constant over a simulation time and between individuals. 
The very first call of an agent is pseudo-randomly generated between $n=0$ and $n=(I + D)/\Delta t$. 

Intensity of sound is not explicitly modelled, but it is assumed that each sound is formed with the same intensity. Sound propagates radially in the form of rings of sound (~\ref{soundprop}), that spreads equally in every directions at the speed of sound at sea level $v_{sound}=340.29 m\times s^{-1}$. Calls that have been generated for a time that exceeds the bat maximum hearing distance (see \ref{subsec: init}) are systematically removed from the sound data storage. 

\begin{figure}[!h]
\centering
\includegraphics{sound_propagation.pdf}
\caption{Schematic representation of sound propagation in the model. The sound spatial distribution at a specific time step $n_t$ for $t\in [n_t\times\Delta t; n_t+1\times\Delta t[$ is defined as $x_{range}$ for $x\in [t_{end}\times\Delta t\times d; t_{start}\times\Delta t\times d[$. $t_{end}$ }
\label{soundprop}
\end{figure}

\subsubsection{Sensory input: 'Sense'}

Hearing is equi-directional.

\subsubsection{Flight pattern: 'Move'}

\subsubsection{Sound reflection and echolocation: 'Echoes', 'Filter'}

\section{Simulations}

\section{Discussion}

What is 'bad': call beam shape and directionality of hearing, sound blocking/shadowing.
What was planned but left out: Adaptation through feedback loop

\section{Conclusion}

\section{Acknowledgements}

\section{References}
\bibliography{Bats2016}{}
\bibliographystyle{Science}

\end{document}